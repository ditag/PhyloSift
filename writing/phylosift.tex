% Template for PLoS
% Version 1.0 January 2009
%
% To compile to pdf, run:
% latex plos.template
% bibtex plos.template
% latex plos.template
% latex plos.template
% dvipdf plos.template

\documentclass[10pt]{article}

% amsmath package, useful for mathematical formulas
\usepackage{amsmath}
% amssymb package, useful for mathematical symbols
\usepackage{amssymb}

% graphicx package, useful for including eps and pdf graphics
% include graphics with the command \includegraphics
\usepackage{graphicx}

% cite package, to clean up citations in the main text. Do not remove.
\usepackage{cite}

\usepackage{color} 

% Use doublespacing - comment out for single spacing
%\usepackage{setspace} 
%\doublespacing


% Text layout
\topmargin 0.0cm
\oddsidemargin 0.5cm
\evensidemargin 0.5cm
\textwidth 16cm 
\textheight 21cm

% Bold the 'Figure #' in the caption and separate it with a period
% Captions will be left justified
\usepackage[labelfont=bf,labelsep=period,justification=raggedright]{caption}

% Use the PLoS provided bibtex style
\bibliographystyle{plos2009}

% Remove brackets from numbering in List of References
\makeatletter
\renewcommand{\@biblabel}[1]{\quad#1.}
\makeatother


% Leave date blank
\date{}

\pagestyle{myheadings}
%% ** EDIT HERE **


%% ** EDIT HERE **
%% PLEASE INCLUDE ALL MACROS BELOW

%% END MACROS SECTION

\begin{document}

% Title must be 150 characters or less
\begin{flushleft}
{\Large
\textbf{Metagenome phylogeny for the 1\%}
}
% Insert Author names, affiliations and corresponding author email.
\\
Person A$^{1}$, 
Person B$^{1,2,3}$, 
Person C$^{1,\ast}$
\\
\bf{1} Genome Center, University of California-Davis, California, United States of America
\\
\bf{2} Department of Evolution and Ecology, University of California-Davis, California, United States of America
\\
\bf{3} Department. of Medical Microbiology and Immunology, University of California-Davis, California, United States of America
\\
$\ast$ E-mail: aarondarling@ucdavis.edu
\end{flushleft}

% Please keep the abstract between 250 and 300 words
\section*{Abstract}
blahdeeblah, blah blah
% Please keep the Author Summary between 150 and 200 words
% Use first person. PLoS ONE authors please skip this step. 
% Author Summary not valid for PLoS ONE submissions.   
\section*{Author Summary}

\section*{Introduction}

The emerging practice of metagenomics has for the first time offered a glimpse of the genetic structure of uncultured microbial communities.
In its current form however, metagenomics destroys some of the most valuable information present in a sample: genetic linkage.
In nearly all metagenomic sample processing methods, cells from the microbial community are lysed together to obtain the DNA.
This practice causes DNA from many different cells to mix together, so that the cellular compartmentalization of individual genotypes is lost.
Long chromosome-scale DNA fragments are then typically sheared by mechanical or enzymatic means into fragments small enough processing with current sequencing chemistries. 
The resulting sequenced fragments are usually less than 1Kbp (though sometimes 10-20kbp). 
The shearing further destroys genetic linkage information, because information on how the short fragments were arranged into chromosome-scale molecules is lost.

Improved sample-processing workflows might preserve the genetic linkage information of a microbial community through the sequencing process.
Single-cell genomics offers an alternative to the standard metagenomics workflow that preserves information about the compartmentalization of genetic material into cells. 
However, single-cell genomics currently presents its own challenges associated with isolating individual cells certain types of samples and amplifying the {DNA} from a single cell to attain the minimum quantity required for sequencing. 
Furthermore, single-cell genomics has extensive equipment requirements and is highly sensitive to contamination by foreign DNA, requiring rigorous laboratory and reagent decontamination.

Given the current limitations of metagenomics, practitioners resort to computational methods to reconstruct the genetic architecture of a microbial community after sequencing.
Computational methods have demonstrated promise for metagenomic analysis and a wide range of approaches have been explored.
Approaches to infer genetic linkage in metagenomes usually address one or more of the following problems:
\begin{itemize}
\item Taxonomic classification of sequences
\item Community structure and organism relative abundance estimation
\item Assigning sequences to bins that correspond to some meaningful group (Binning)
\end{itemize}
Each of these problems are inter-related and many methods address two or more of these at once.
\textit{Question: should we define each of these problems?}

Molecular evolution, the joint processes of reproduction with mutation and natural selection acting on mutations, provides the primary signal for most read classification methods.
As species diverge, differences such as nucleotide substitutions, insertions, deletions, and genomic rearrangements accumulate in their genomes, and these differences can be leveraged to infer a sequence's origin in a mixed metagenomic sample.

In the present manuscript, we introduce a new method for reconstructing the genetic architecture of a metagenomic sample and for comparison of community structure among multiple related samples.
The new method leverages explicitly phylogenetic models of molecular evolution. 
We contribute an open-source implementation of the method that has been engineered for ease-of-use on 64-bit Linux and Mac platforms.
Finally, we compare the features and performance of the new method to some related methods to provide readers with insight into when use of the new method is and is not appropriate.


\subsection*{Previous work}
Metagenomics is a burgeoning field, we focus here on shotgun metagenome sequencing approaches and 
do not discuss analysis methods for other aspects of microbial ecology such as amplicon sequencing data such as QIIME and others. 
\subsubsection*{Whole metagenome binning}
- composition classifiers
  - TACOA~\cite{Diaz2009}, PhyloPythia 1 \& 2\cite{Patil2011},Eu-Detect\cite{Mohammed2011},ProViDE~\cite{Ghosh2011}
- identity/homology classifiers
  - MEGAN~\cite{Huson2007}, SORT-Items~\cite{Haque2009}, NBC~\cite{Rosen2011}, MTR~\cite{Gori2011}, others
\subsubsection*{Estimating community composition from metagenomes}
      - AMPHORA\cite{WuEisen2008}, MLTreeMap\cite{Stark2010}, others?
      - MetaPHlan (under review)

% You may title this section "Methods" or "Models". 
% "Models" is not a valid title for PLoS ONE authors. However, PLoS ONE
% authors may use "Analysis" 
\section*{Methods or Design and Implementation}

\subsection*{Detailed PhyloSift workflow}
Stream computing
\subsubsection*{Sequence identity search}
Process sequence (reads or assembly) to identify similarity to marker genes
      - Evaluated BLAST~\cite{Altschul1997}, BLAST+~\cite{Camacho2009}, RAPsearch2~\cite{Zhao2011}, LAST~\cite{Kiełbasa2011}, bowtie2~\cite{Langmead2009}. Picked LAST, bowtie2.
\subsubsection*{Alignment to reference multiple alignment}
      - reverse translation to DNA when possible
\subsubsection*{Placement on a phylogenetic reference tree}

\subsubsection*{Taxonomic summary of read placements}
      - taxonomic summary
      - other summaries
\subsubsection*{Comparison among samples}

\subsection*{PhyloSift database update workflow}
\subsubsection*{Acquire new genome data}
\subsubsection*{Run PhyloSift workflow on each genome}
\subsubsection*{Taxonomic reconciliation}

% Results and Discussion can be combined.
\section*{Results}


\subsection*{look Ma, I'm famous}

\subsection*{Simulated data}
    - How they were generated
    - Building PhyloSift DBs without test datasets
\subsubsection*{Precision, recall, and F1}
\subsubsection*{Relationship between F1 and neighbor taxon density}

\subsubsection*{Community structure accuracy}
RMSD or other metric?

\subsubsection*{Accuracy on particular taxonomic groups}
Bacteria, Arch, Eukarya, viral

\subsection*{Application to a real dataset}
 (which one?)

\subsection*{PhyloSift crushes its former self}


\section*{Discussion}
  - Fundamental limits to computational methods -- resolving linkage among polymorphisms in a population
  - Advantages to the approach taken by PhyloSift
  - Disadvantages
    - many


\subsection*{Limitations and scope}


\subsection*{Future work}


\section*{Availability}
Software for Linux and Mac OS X, along with source code is freely available from http://github.com/gjospin/PhyloSift
The source code has been licensed under the GNU Public License (GPL) v3.0.

% Do NOT remove this, even if you are not including acknowledgments
\section*{Acknowledgments}
Funding from Dept. of Homeland Security, DOE.

%\section*{References}
% The bibtex filename
\bibliography{phylosift}

\clearpage

\section*{Figure Legends}
%\begin{figure}[!ht]
%\begin{center}
%%\includegraphics[width=4in]{figure_name.2.eps}
%\end{center}
%\caption{
%{\bf Bold the first sentence.}  Rest of figure 2  caption.  Caption 
%should be left justified, as specified by the options to the caption 
%package.
%}
%\label{Figure_label}
%\end{figure}
\begin{figure}[hp]
\begin{center}
\end{center}
\caption{\textbf{A really pretty picture.} As you can see, the curves in this image are stunning.
Pick your jaw up off the floor, this ain't no joke.}
\label{fig:the_first_figure}
\end{figure}

\clearpage

\section*{Tables}


\clearpage

\end{document}

